% latex FHNeqn.tex
% dvips -E -o FHNeqn.eps FHNeqn.dvi
%

\documentclass[12pt]{article}

 \usepackage[dvips]{graphicx}

%% The amssymb package provides various useful mathematical symbols
 \usepackage{amssymb}
  
  \pagestyle{empty}
  \thispagestyle{empty} 
  \setlength{\parindent}{0mm}


  \begin{document}

\vspace{0.5cm}

   $\displaystyle \frac{d \phi_e}{dt} = \dot{\phi_e} $, \\
   
\vspace{0.5cm}
  
   $\displaystyle \frac{d \dot{\phi_e}}{dt} =   \gamma_e^2 (S(V_e) - \phi_e) - 2 \gamma_e \dot{\phi_e}  + v_e^2 \nabla^2 \phi_e $, \\

\vspace{0.5cm}

   $\displaystyle \frac{d V_e}{dt}= \dot{V_e} $, \\

\vspace{0.5cm}

   $\displaystyle \frac{d \dot{V_e}}{dt} =  \alpha \beta (\nu_{ee}\phi_e + \nu_{ei} S(V_i) + \nu_{es} S(V_s(t-\tau)) - V_e) - ( \alpha + \beta) \dot{V_e}$, \\

\vspace{0.5cm}

   $\displaystyle \frac{d V_s}{dt}= \dot{V_s}$, \\

\vspace{0.5cm}

   $\displaystyle \frac{d \dot{V_s}}{dt} =  \alpha \beta (\nu_{sn}\phi_n + \nu_{sr} S(V_r) + \nu_{se}\phi_e(t-\tau) - V_s)-( \alpha + \beta) \dot{V_s}$, \\

\vspace{0.5cm}

   $\displaystyle \frac{d V_r}{dt} = \dot{V_r}$, \\

\vspace{0.5cm}

   $\displaystyle \frac{d \dot{V_r}}{dt} =  \alpha \beta (\nu_{re}\phi_e(t-\tau) + \nu_{rs} S(V_s) - V_r)-( \alpha + \beta) \dot{V_r}$, \\

%\vspace{1cm}
%BUT THINK IT SHOULD BE: \\
%$\displaystyle \frac{d \dot{\phi_e}}{dt} = -\phi_e + S(V_e) \gamma^2 - 2 \gamma \dot{\phi_e} +  \nabla^2 \phi_e / r_e^2 $,
%
%\vspace{1cm}
%v^2 \nabla^2 \phi_e
%Where, $dx$ is distance between points and $\nabla^2 $ is the discrete laplacian???...

 


  \end{document}
%%%EoF%%%
